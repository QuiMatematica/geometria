\chapter{Esercizi e quiz}
\label{ch:quiz}

%\section{Dall'eserciziario dello stage senior}
%\label{sec:quiz_stage_senior}

\section{Quiz dalle gare distrettuali}
\label{sec:quiz_gare_distrettuali}

\begin{esercizio}[Gare distrettuail 2023, n. 3]
    \label{ex:distrettuale_2023_3}

    \begin{wrapfigure}{r}{0.25\textwidth}
        \definecolor{cqcqcq}{rgb}{0.7529411764705882,0.7529411764705882,0.7529411764705882}
        \definecolor{uququq}{rgb}{0.25098039215686274,0.25098039215686274,0.25098039215686274}
        \begin{tikzpicture}[line cap=round,line join=round,>=triangle 45,x=1cm,y=1cm]
        \clip(-.5,-1) rectangle (4.5,2.3);
        \draw [line width=1pt,color=cqcqcq,fill=cqcqcq,fill opacity=1] (1,1) circle (1cm);
        \draw [line width=1pt,color=cqcqcq,fill=cqcqcq,fill opacity=1] (3,1) circle (1cm);
        \fill[line width=1pt,color=cqcqcq,fill=cqcqcq,fill opacity=1] (3,1) -- (1,1) -- (2,-0.7320508075688773) -- cycle;
        \fill[line width=1pt,color=cqcqcq,fill=cqcqcq,fill opacity=1] (2,-0.7320508075688773) -- (0.5,0.13397459621556135) -- (1,1) -- cycle;
        \fill[line width=1pt,color=cqcqcq,fill=cqcqcq,fill opacity=1] (2,-0.7320508075688773) -- (3.5,0.13397459621555896) -- (3,1) -- cycle;
        \draw [line width=1pt,color=cqcqcq] (3,1)-- (1,1);
        \draw [line width=1pt,color=cqcqcq] (1,1)-- (2,-0.7320508075688773);
        \draw [line width=1pt,color=cqcqcq] (2,-0.7320508075688773)-- (3,1);
        \draw [line width=1pt,color=cqcqcq] (2,-0.7320508075688773)-- (0.5,0.13397459621556135);
        \draw [line width=1pt,color=cqcqcq] (0.5,0.13397459621556135)-- (1,1);
        \draw [line width=1pt,color=cqcqcq] (1,1)-- (2,-0.7320508075688773);
        \draw [line width=1pt,color=cqcqcq] (2,-0.7320508075688773)-- (3.5,0.13397459621555896);
        \draw [line width=1pt,color=cqcqcq] (3.5,0.13397459621555896)-- (3,1);
        \draw [line width=1pt,color=cqcqcq] (3,1)-- (2,-0.7320508075688773);
        \draw [line width=1pt] (1,1) circle (1cm);
        \draw [line width=1pt] (3,1) circle (1cm);
        \draw [line width=1pt] (2,-0.7320508075688773)-- (0.5,0.13397459621556135);
        \draw [line width=1pt] (2,-0.7320508075688773)-- (3.5,0.13397459621555896);
        \draw [line width=.8pt,dash pattern=on 2pt off 2pt] (2,-5.161567540068593) -- (2,4.761305273089394);
        \begin{scriptsize}
        \draw [fill=black] (1,1) circle (1.5pt);
        \draw[color=black] (1.0155133167916046,1.368724899184351) node {$O_1$};
        \draw [fill=black] (3,1) circle (1.5pt);
        \draw[color=black] (3.0091498637274614,1.3800523795646684) node {$O_2$};
        \draw [fill=black] (2,-0.7320508075688773) circle (1.5pt);
        \draw[color=black] (2.1482613548233416,-0.8288062945972126) node {$P$};
        \end{scriptsize}
        \end{tikzpicture}
    \end{wrapfigure}

    Alice disegna un cuore sul quaderno di matematica come segue: prima disegna due circonferenze di raggio 1 cm e centri
    $O_1$, $O_2$, tangenti esternamente.
    Chiamata $r$ la tangente comune alle due circonferenze passante per il punto di contatto, sceglie poi un punto $P$
    su $r$ in modo che si abbia $O_1\hat{P}O_2 = 60^\circ$ e traccia le tangenti alle due circonferenze passanti per
    $P$.
    Quanto vale l'area del cuore ottenuto (ovvero l'area ombreggiata in figura) in cm$^2$.

    (A) $2\sqrt{3} + \frac{4}{3}\pi$ \quad (B) $\sqrt{3} + \pi$ \quad (C) $\frac{7}{3}\pi$ \quad
    (D) $\frac{3}{2} + 2\pi$ \quad

    (E)~Nessuna~delle~precedenti
\end{esercizio}

\begin{esercizio}[Gare distrettuali 2023, n. 10]
    \label{ex:distrettuale_2023_10}
    Sia $ABCD$ un parallelogramma tale che la bisettrice uscente da $B$ interseca il lato $CD$ nel suo punto medio $M$.
    Il lato $BC$ è lungo 6, e la diagonale $AC$ è lunga 14.
    Determinare la lunghezza di $AM$.

    (A) $2\sqrt {19}$ \quad
    (B) $14 - 3\sqrt{3}$ \quad
    (C) $4\sqrt{5}$ \quad
    (D) $9$ \quad
    (E) $2\sqrt{22}$
\end{esercizio}

%\section{Altri quiz}
%\label{sec:quiz_altri}

