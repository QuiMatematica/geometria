\chapter{Soluzioni}
\label{ch:soluzioni}

\begin{soluzione}{ex:distrettuale_2023_10}

\definecolor{zzttqq}{rgb}{0.6,0.2,0}
\definecolor{uuuuuu}{rgb}{0.26666666666666666,0.26666666666666666,0.26666666666666666}

\begin{tikzpicture}[line cap=round,line join=round,>=triangle 45,x=.4cm,y=.4cm]
\clip(-1,-1) rectangle (14,8);
\fill[line width=1pt,color=zzttqq,fill=zzttqq,fill opacity=0.10000000149011612] (0,0) -- (12,0) -- (12.666666666666666,5.962847939999441) -- (0.6666666666666661,5.962847939999441) -- cycle;
\draw [line width=1pt] (0,0)-- (12,0);
\draw [line width=1pt] (0,0)-- (12.666666666666666,5.962847939999441);
\draw [line width=1pt] (0,0)-- (6.666666666666665,5.962847939999441);
%\draw [line width=1pt] (0,0)-- (1.3333333333333317,11.925695879998878);
%\draw [line width=1pt] (1.3333333333333317,11.925695879998878)-- (13.333333333333332,11.92569587999888);
%\draw [line width=1pt] (13.333333333333332,11.92569587999888)-- (12,0);
\draw [line width=1pt] (0.6666666666666661,5.962847939999441)-- (12.666666666666666,5.962847939999441);
\draw [line width=1pt] (12,0)-- (6.666666666666665,5.962847939999441);
\draw [line width=1pt,color=zzttqq] (0,0)-- (12,0);
\draw [line width=1pt,color=zzttqq] (12,0)-- (12.666666666666666,5.962847939999441);
\draw [line width=1pt,color=zzttqq] (12.666666666666666,5.962847939999441)-- (0.6666666666666661,5.962847939999441);
\draw [line width=1pt,color=zzttqq] (0.6666666666666661,5.962847939999441)-- (0,0);
\begin{scriptsize}
\draw [fill=uuuuuu] (0,0) circle (1pt);
\draw[color=black] (-0.5103139333868644,-0.09958656484186007) node {$A$};
\draw [fill=uuuuuu] (12,0) circle (1pt);
\draw[color=black] (12.54516265250897,0.1789302689905845) node {$B$};
\draw [fill=uuuuuu] (12.666666666666666,5.962847939999441) circle (1pt);
\draw[color=black] (13.137010924402915,6.132227592159087) node {$C$};
\draw [fill=uuuuuu] (0.6666666666666661,5.962847939999441) circle (1pt);
\draw[color=black] (0.2207927554233024,6.758890468282088) node {$D$};
\draw [fill=uuuuuu] (6.666666666666665,5.962847939999441) circle (1pt);
\draw[color=black] (6.59186532934047,6.793705072511144) node {$M$};
\end{scriptsize}
\end{tikzpicture}

    \textbf{Dati:}

    $ABCD$ parallelogramma

    $BM$ bisettrice di $\hat{ABC}$

    $M$ punto medio di $CD$

    $\overline{BC} = 6$

    $\overline{AC} = 14$

    \textbf{Richiesta:}

    $\overline{AM}$

    \bigskip
    Innanzitutto una considerazione sulla figura.
    L'angolo interno al parallelogramma $B$ potrebbe essere acuto od ottuso.
    A un certo punto della soluzione risulterà evidente che tale angolo deve essere ottuso.

    Gli angoli $A\hat{B}M$ e $\hat{BMC}$ sono alterni interni per le parallele $AB$ e $CD$ tagliate dalla
    trasversale $BM$;
    quindi sono congruenti.
    Pertanto per il teorema inverso dei triangoli isosceli, i segmenti $BC$ e $CM$ sono congruenti:
    $\overline{BC} = \overline{CM} = 6$.

    $M$ è il punto medio di $CD$, quindi $\overline{CD} = 2 \overline{CM} = 2 \cdot 6 = 12$.
    Come pure $\overline{AB} = \overline{CD} = 12$.

    Chiamo: $\alpha = \hat{DAB}$, $\beta = \hat{ABC}$.
    Dal momento che $ABCD$ è un parallelogramma, risulta $\alpha + \beta = \pi$.
    Lo vediamo anche nel triangolo isoscele $BCM$, dove i due angoli congruenti sono pari a $\frac{\beta}{2}$, mentre
    il terzo angolo interno è $\alpha$.

    Anche il triangolo $ADM$ è isoscele, perché $\overline{AD} = \overline{DM} = 6$.
    L'angolo interno in $D$ è $\beta$, quindi gli angoli alla base sono $\frac{\alpha}{2}$.

    Determino ora l'ampiezza dell'angolo $\hat{AMB}$, partendo dall'angolo piatto $\hat{CMD}$.
    \begin{align*}
        \hat{AMB} &= \hat{CMD} - \hat{CMB} - \hat{AMD} = \\
        &= \pi - \frac{\beta}{2} - \frac{\alpha}{2} = \\
        &= \pi - \biggl(\frac{\beta}{2} + \frac{\alpha}{2} \biggr) = \\
        &= \pi - \frac{\pi}{2} = \\
        &= \frac{\pi}{2}
    \end{align*}

    Quindi il triangolo $AMB$ è rettangolo.

    Ora utilizzo il teorema del coseno per ottenere informazioni sull'angolo interno $\hat{ABC} = \beta$:
    \begin{gather*}
        \overline{AB}^2 + \overline{BC}^2 - 2 \cdot \overline{AB} \cdot \overline{BC} \cdot \cos \beta = \overline{AC}^2 \\
        12^2 + 6^2 - 2 \cdot 12 \cdot 6 \cdot \cos \beta = 14^2 \\
        144 + 36 - 144 \cos \beta = 196 \\
        -144 \cos \beta = 16 \\
        \cos \beta = -\dfrac{16}{144} \\
        \cos \beta = -\dfrac{1}{9}
    \end{gather*}

    Ora posso calcolare $\overline{AM}$ in quanto cateto del triangolo rettangolo $AMB$, di cui conosco l'ipotenusa
    $\overline{AB} = 12$ e l'angolo $\hat{ABM} = \frac{\beta}{2}$.
    Avrò bisogno delle formule di bisezione.
    \begin{align*}
        \overline{AM} &= \overline{AB} \cdot \sin \frac{\beta}{2} = \\
        &= \overline{AB} \cdot \sqrt {\dfrac{1 - \cos \beta}{2}} = \\
        &= 12 \cdot \sqrt {\dfrac{1 + \frac{1}{9}}{2}} = \\
        &= 12 \cdot \sqrt {\dfrac{10}{9} \cdot \dfrac{1}{2}} = \\
        &= 12 \cdot \sqrt {\dfrac{5}{9}} = \\
        &= \dfrac{12}{3} \sqrt {5} = \\
        &= 4 \sqrt {5}
    \end{align*}

    Risposta \textbf{C}.
\end{soluzione}