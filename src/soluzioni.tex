\chapter{Soluzioni}
\label{ch:soluzioni}

\begin{soluzione}{ex:distrettuale_2023_3}
    \begin{wrapfigure}{l}{0.5\textwidth}
        \definecolor{cqcqcq}{rgb}{0.7529411764705882,0.7529411764705882,0.7529411764705882}
        \definecolor{uququq}{rgb}{0.25098039215686274,0.25098039215686274,0.25098039215686274}
        \begin{tikzpicture}[line cap=round,line join=round,>=triangle 45,x=1cm,y=1cm]
        \clip(-.5,-1) rectangle (4.5,2.3);
        \draw [line width=1pt,color=cqcqcq,fill=cqcqcq,fill opacity=1] (1,1) circle (1cm);
        \draw [line width=1pt,color=cqcqcq,fill=cqcqcq,fill opacity=1] (3,1) circle (1cm);
        \fill[line width=1pt,color=cqcqcq,fill=cqcqcq,fill opacity=1] (3,1) -- (1,1) -- (2,-0.7320508075688773) -- cycle;
        \fill[line width=1pt,color=cqcqcq,fill=cqcqcq,fill opacity=1] (2,-0.7320508075688773) -- (0.5,0.13397459621556135) -- (1,1) -- cycle;
        \fill[line width=1pt,color=cqcqcq,fill=cqcqcq,fill opacity=1] (2,-0.7320508075688773) -- (3.5,0.13397459621555896) -- (3,1) -- cycle;
        \draw [line width=1pt,color=cqcqcq] (3,1)-- (1,1);
        \draw [line width=1pt,color=cqcqcq] (1,1)-- (2,-0.7320508075688773);
        \draw [line width=1pt,color=cqcqcq] (2,-0.7320508075688773)-- (3,1);
        \draw [line width=1pt,color=cqcqcq] (2,-0.7320508075688773)-- (0.5,0.13397459621556135);
        \draw [line width=1pt,color=cqcqcq] (0.5,0.13397459621556135)-- (1,1);
        \draw [line width=1pt,color=cqcqcq] (1,1)-- (2,-0.7320508075688773);
        \draw [line width=1pt,color=cqcqcq] (2,-0.7320508075688773)-- (3.5,0.13397459621555896);
        \draw [line width=1pt,color=cqcqcq] (3.5,0.13397459621555896)-- (3,1);
        \draw [line width=1pt,color=cqcqcq] (3,1)-- (2,-0.7320508075688773);
        \draw [line width=1pt] (1,1) circle (1cm);
        \draw [line width=1pt] (3,1) circle (1cm);
        \draw [line width=1pt] (2,-0.7320508075688773)-- (0.5,0.13397459621556135);
        \draw [line width=1pt] (2,-0.7320508075688773)-- (3.5,0.13397459621555896);
        \draw [line width=.8pt,dash pattern=on 2pt off 2pt] (2,-5.161567540068593) -- (2,4.761305273089394);
        \draw [line width=1pt] (0.5,0.13397459621556135)-- (1,1);
        \draw [line width=1pt] (3.5,0.13397459621555896)-- (3,1);
        \draw [line width=1pt] (1,1)-- (3,1);
        \draw [line width=1pt] (2,-0.7320508075688773)-- (3,1);
        \draw [line width=1pt] (1,1)-- (2,-0.7320508075688773);
        \begin{scriptsize}
        \draw [fill=black] (1,1) circle (1.5pt);
        \draw[color=black] (1.0155133167916046,1.368724899184351) node {$O_1$};
        \draw [fill=black] (3,1) circle (1.5pt);
        \draw[color=black] (3.0091498637274614,1.3800523795646684) node {$O_2$};
        \draw [fill=black] (2,-0.7320508075688773) circle (1.5pt);
        \draw[color=black] (2.1482613548233416,-0.8288062945972126) node {$P$};
        \draw [fill=black] (0.5,0.13397459621556135) circle (1.5pt);
        \draw[color=black] (0.3,-.1) node {$A$};
        \draw [fill=black] (3.5,0.13397459621555896) circle (1.5pt);
        \draw[color=black] (3.7,-.1) node {$C$};
        \draw [fill=black] (2,1) circle (1.5pt);
        \draw[color=black] (2.2,1.2) node {$B$};
        \end{scriptsize}
        \end{tikzpicture}
    \end{wrapfigure}

    \textbf{Dati:}

    $\overline{O_1A} = \overline{O_2A} = 1 \text{cm}$

    $O_1\hat{P}O_2 = 60^\circ$

    \textbf{Richiesta:}

    Area ombreggiata

    \bigskip
    Tenendo presente che le due circonferenze sono congruenti (hanno infatti lo stesso raggio), per il teorema
    delle tangenti sappiamo che $PA \cong PB \cong PC$ e $PO_1 \cong PO_2$.
    E i triangoli $PAO_1$, $PBO_1$, $PBO_2$ e $PC=_2$ sono tutti congruenti (oltre che rettangoli).

    Dal momento che $PO_1 \cong PO_2$, il triangolo $O_1 P O_2$ è isoscele.
    Inoltre l'angolo $O_1\hat{P}O_2$ ha ampiezza $60^\circ$, quindi $O_1 P O_2$ è equilatero, di lato 2.
    L'altezza $PB$ di questo triangolo è lunga $\sqrt{3}$.
    
    L'area del quadrilatero $PAO_1 O_2 C$ può essere calcolata come doppio dell'area del triangolo $O_1 P O_2$:
    \begin{equation*}
        A_{PAO_1 O_2 C} = 2 A_{O_1 P O_2} = 2 \cdot \dfrac{\overline{O_1 O_2} \cdot \overline{PB}}{2} = 
        2 \cdot \dfrac{2 \cdot \sqrt{3}}{2} = 2\sqrt {3}
    \end{equation*}

    Non rimane che calcolare le aree dei settori circolari che completano la figura.

    Visto che il triangolo $O_1 P O_2$ è equilatero, l'angolo $PO_1 B$ ha ampiezza $60^\circ$.
    Quindi, per congruenza, anche l'angolo $P O_1 A$ ha ampiezza $60^\circ$.
    Quindi $AO_1 B = 120^\circ$.
    Ciascun settore circolare, quindi, è $\frac{2}{3}$ dell'intero cerchio.

    L'area del cerchio (di raggio 1) è $\pi$, quindi due settori circolari pari a $\frac{2}{3}$ del cerchio hanno
    in totale area $\frac{4}{3}\pi$.

    Quindi l'area totale è $\frac{4}{3}\pi + 2\sqrt {3}$.

   Risposta \textbf{A}.
\end{soluzione}

\begin{soluzione}{ex:distrettuale_2023_10}

    \begin{wrapfigure}{l}{0.5\textwidth}
        \definecolor{zzttqq}{rgb}{0.6,0.2,0}
        \definecolor{uuuuuu}{rgb}{0.26666666666666666,0.26666666666666666,0.26666666666666666}
        \begin{tikzpicture}[line cap=round,line join=round,>=triangle 45,x=.4cm,y=.4cm]
        \clip(-1,-.4) rectangle (14,7);
        \fill[line width=1pt,color=zzttqq,fill=zzttqq,fill opacity=0.10000000149011612] (0,0) -- (12,0) -- (12.666666666666666,5.962847939999441) -- (0.6666666666666661,5.962847939999441) -- cycle;
        \draw [line width=1pt] (0,0)-- (12,0);
        \draw [line width=1pt] (0,0)-- (12.666666666666666,5.962847939999441);
        \draw [line width=1pt] (0,0)-- (6.666666666666665,5.962847939999441);
        %\draw [line width=1pt] (0,0)-- (1.3333333333333317,11.925695879998878);
        %\draw [line width=1pt] (1.3333333333333317,11.925695879998878)-- (13.333333333333332,11.92569587999888);
        %\draw [line width=1pt] (13.333333333333332,11.92569587999888)-- (12,0);
        \draw [line width=1pt] (0.6666666666666661,5.962847939999441)-- (12.666666666666666,5.962847939999441);
        \draw [line width=1pt] (12,0)-- (6.666666666666665,5.962847939999441);
        \draw [line width=1pt,color=zzttqq] (0,0)-- (12,0);
        \draw [line width=1pt,color=zzttqq] (12,0)-- (12.666666666666666,5.962847939999441);
        \draw [line width=1pt,color=zzttqq] (12.666666666666666,5.962847939999441)-- (0.6666666666666661,5.962847939999441);
        \draw [line width=1pt,color=zzttqq] (0.6666666666666661,5.962847939999441)-- (0,0);
        \begin{scriptsize}
        \draw [fill=uuuuuu] (0,0) circle (1pt);
        \draw[color=black] (-0.5103139333868644,-0.09958656484186007) node {$A$};
        \draw [fill=uuuuuu] (12,0) circle (1pt);
        \draw[color=black] (12.54516265250897,0.1789302689905845) node {$B$};
        \draw [fill=uuuuuu] (12.666666666666666,5.962847939999441) circle (1pt);
        \draw[color=black] (13.137010924402915,6.132227592159087) node {$C$};
        \draw [fill=uuuuuu] (0.6666666666666661,5.962847939999441) circle (1pt);
        \draw[color=black] (0.2207927554233024,6.758890468282088) node {$D$};
        \draw [fill=uuuuuu] (6.666666666666665,5.962847939999441) circle (1pt);
        \draw[color=black] (6.59186532934047,6.793705072511144) node {$M$};
        \end{scriptsize}
        \end{tikzpicture}
    \end{wrapfigure}

    \textbf{Dati:}

    $ABCD$ parallelogramma

    $BM$ bisettrice di $\hat{ABC}$

    $M$ punto medio di $CD$

    $\overline{BC} = 6$

    $\overline{AC} = 14$

    \textbf{Richiesta:}

    $\overline{AM}$

    \bigskip
    Innanzitutto una considerazione sulla figura.
    L'angolo interno al parallelogramma $B$ potrebbe essere acuto od ottuso.
    A un certo punto della soluzione risulterà evidente che tale angolo deve essere ottuso.

    Gli angoli $A\hat{B}M$ e $\hat{BMC}$ sono alterni interni per le parallele $AB$ e $CD$ tagliate dalla
    trasversale $BM$;
    quindi sono congruenti.
    Pertanto per il teorema inverso dei triangoli isosceli, i segmenti $BC$ e $CM$ sono congruenti:
    $\overline{BC} = \overline{CM} = 6$.

    $M$ è il punto medio di $CD$, quindi $\overline{CD} = 2 \overline{CM} = 2 \cdot 6 = 12$.
    Come pure $\overline{AB} = \overline{CD} = 12$.

    Chiamo: $\alpha = \hat{DAB}$, $\beta = \hat{ABC}$.
    Dal momento che $ABCD$ è un parallelogramma, risulta $\alpha + \beta = \pi$.
    Lo vediamo anche nel triangolo isoscele $BCM$, dove i due angoli congruenti sono pari a $\frac{\beta}{2}$, mentre
    il terzo angolo interno è $\alpha$.

    Anche il triangolo $ADM$ è isoscele, perché $\overline{AD} = \overline{DM} = 6$.
    L'angolo interno in $D$ è $\beta$, quindi gli angoli alla base sono $\frac{\alpha}{2}$.

    Determino ora l'ampiezza dell'angolo $\hat{AMB}$, partendo dall'angolo piatto $\hat{CMD}$.
    \begin{align*}
        \hat{AMB} &= \hat{CMD} - \hat{CMB} - \hat{AMD} = \\
        &= \pi - \frac{\beta}{2} - \frac{\alpha}{2} = \\
        &= \pi - \biggl(\frac{\beta}{2} + \frac{\alpha}{2} \biggr) = \\
        &= \pi - \frac{\pi}{2} = \\
        &= \frac{\pi}{2}
    \end{align*}

    Quindi il triangolo $AMB$ è rettangolo.

    Seguono quattro soluzioni diverse:
    nella prima si utilizza la trigonometria, nella seconda la formula di Erone e un'equazione biquadratica,
    nella terza ancora la formula di Erone e un sistema simmetrico,
    nella quarta invece si utilizza il teorema di Pitagora e un semplice sistema di quarto grado.

    \bigskip
    \textbf{Prima soluzione.}

    Ora utilizzo il teorema del coseno per ottenere informazioni sull'angolo interno $\hat{ABC} = \beta$:
    \begin{gather*}
        \overline{AB}^2 + \overline{BC}^2 - 2 \cdot \overline{AB} \cdot \overline{BC} \cdot \cos \beta = \overline{AC}^2 \\
        12^2 + 6^2 - 2 \cdot 12 \cdot 6 \cdot \cos \beta = 14^2 \\
        144 + 36 - 144 \cos \beta = 196 \\
        -144 \cos \beta = 16 \\
        \cos \beta = -\dfrac{16}{144} \\
        \cos \beta = -\dfrac{1}{9}
    \end{gather*}

    Ora posso calcolare $\overline{AM}$ in quanto cateto del triangolo rettangolo $AMB$, di cui conosco l'ipotenusa
    $\overline{AB} = 12$ e l'angolo $\hat{ABM} = \frac{\beta}{2}$.
    Avrò bisogno delle formule di bisezione.
    \begin{align*}
        \overline{AM} &= \overline{AB} \cdot \sin \frac{\beta}{2} = \\
        &= \overline{AB} \cdot \sqrt {\dfrac{1 - \cos \beta}{2}} = \\
        &= 12 \cdot \sqrt {\dfrac{1 + \frac{1}{9}}{2}} = \\
        &= 12 \cdot \sqrt {\dfrac{10}{9} \cdot \dfrac{1}{2}} = \\
        &= 12 \cdot \sqrt {\dfrac{5}{9}} = \\
        &= \dfrac{12}{3} \sqrt {5} = \\
        &= 4 \sqrt {5}
    \end{align*}

    Risposta \textbf{C}.

    \bigskip
    \textbf{Seconda soluzione, proposta da Francesco P.}

    Chiamo $N$ il punto medio del lato $AB$ (non è segnato sulla figura).

    $A_{ABCD} = 2A_{ANMD}$ poiché $A_{ANMD} + A_{BCMN} = A_{ABCD}$ e $ANMD \cong BCMN$.

    $A_{ABCD} = 2A_{ABC}$ poiché $ABC \cong ACD$ e $A_{ABC} + A_{ACD} = A_{ABCD}$.
    
    Allora $A_{ANMD} = A_{ABC}$, e poiché $A_{AMD} = \frac{1}{2}A_{ANMD}$, si ha $A_{ABC} = 2A_{ADM}$.
    
    Chiamo $\overline{AM} = x > 0$.
    Utilizzando la formula di Erone:
    \begin{align*}
        A_{ABC} &= \sqrt {p(p - AB)(p-BC)(p-AC)} = \\
        &= \sqrt{16(16-12)(16-6)(16-14)} = \\
        &= \sqrt {16 \cdot 4 \cdot 10 \cdot 2} = \\
        &= 16 \sqrt {5} \\
        A_{ADM}
        &= \sqrt { \dfrac{\scriptstyle \overline{AD} + \overline{DM} + \overline{AM}}{\scriptstyle 2} \cdot
        \dfrac{\scriptstyle \overline{AD} + \overline{DM} - \overline{AM}}{\scriptstyle 2} \cdot
        \dfrac{\scriptstyle \overline{AD} - \overline{DM} + \overline{AM}}{\scriptstyle 2} \cdot
        \dfrac{\scriptstyle -\overline{AD} + \overline{DM} + \overline{AM}}{\scriptstyle 2}} = \\
        &= \sqrt {\dfrac{12 + x}{2}  \cdot \dfrac{12 - x}{2}  \cdot \dfrac{x}{2}  \cdot \dfrac{x}{2}} = \\
        &= \dfrac{x}{4}\sqrt {144 - x^2}
    \end{align*}

    Risolviamo quindi l'equazione:
    \begin{gather*}
        16 \sqrt {5} = \dfrac{x}{4}\sqrt {144 - x^2} \\
        1280 = \dfrac{x^2}{16} \cdot (144 - x^2) \\
        5120 = x^2 \cdot (144 - x^2) \\
        x^4 - 144 x^2 + 5120 = 0 \\
        \dfrac{\Delta}{4} = 144^2 - 5120 = 5184 - 5120 = 64 \\
        x^2 = -72 \pm \sqrt {64} = 72 \pm 8 \\
        x^2 = 64 \lor x^2 = 80 \\
        x = 8 \lor x = 4\sqrt {5}
    \end{gather*}

    $AM$ è la diagonale maggiore di $ANMD$, quindi $\overline{AM} = 4\sqrt {5}$ e $\overline{DN} = 8$.

    \bigskip
    \textbf{Terza soluzione.}

    Chiamiamo $\overline{AM} = x$ e $\overline{BM} = y$.
    I triangoli $ABC$ e $ABM$ sono equivalenti in quanto hanno la base in comune (il lato $AB$) e l'altezza in comune
    (l'altezza del trapezio).
    Possiamo calcolare l'area di $ABC$ con la formula di Erone (come nella soluzione precedente) e l'area di $ABM$
    utilizzando i cateti (dal momento che $A\hat{M}B = \pi$).

    Dobbiamo dunque risolvere il seguente sistema simmetrico:
    \begin{gather*}
        \begin{cases}
            \overline{AM}^2 + \overline{BM}^2 = \overline{AB}^2 \\
            \dfrac{\overline{AM} \cdot \overline{BM}}{2} = A_{ABC0}
        \end{cases}
        \\
        \begin{cases}
            x^2 + y^2 = 12^2 \\
            \dfrac{xy}{2} = 16\sqrt {5}
        \end{cases}
        \\
        \begin{cases}
            (x + y)^2 - 2xy = 144 \\
            xy = 32\sqrt{5}
        \end{cases}
        \\
        \begin{cases}
            (x+y)^2 = 144 + 64\sqrt {5} \\
            xy = 32\sqrt {5}
        \end{cases}
        \\
        \begin{cases}
            x + y = 8 + 4\sqrt {5} \\
            xy = 32\sqrt {5}
        \end{cases}
        \\
        (x, y) = (8, 4\sqrt {5}) \lor (x,y) = (4\sqrt {5}, 8)
    \end{gather*}

    Considerando che $AM > BM$ si ha che $\overline{AM} = 4\sqrt {5}$.

    \bigskip
    \textbf{Quarta soluzione.}
    Chiamiamo $H$ la proiezione del punto $C$ sulla retta $AB$ (non è segnata sul disegno).
    Otteniamo così due triangoli rettangoli: $AHC$ e $BHC$.

    Chiamiamo $\overline{BH} = b$ e $\overline{CH} = h$.
    Possiamo così applicare il teorema di Pitagora ai due triangoli rettangoli e trovare le lunghezze $b$ e $h$.
    \begin{gather*}
        \begin{cases}
            \overline{AH}^2 + \overline{CH}^2 = \overline{BC}^2 \\
            \overline{BH}^2 + \overline{CH}^2 = \overline{AC}^2
        \end{cases}
        \\
        \begin{cases}
            (12 + b)^2 + h^2 = 14^2 \\
            b^2 + h^2 = 6^2
        \end{cases}
        \\
        \begin{cases}
            144 + 2b + b^2 + h^2 = 196 \\
            b^2 + h^2 = 36
        \end{cases}
    \end{gather*}

    Sottraendo membro a membro si ottiene:
    \begin{gather*}
        144 + 24b = 160 \\
        24b = 16 \\
        b = \dfrac{2}{3}
    \end{gather*}
    
    Tornando alla seconda equazione si ha:
    \begin{gather*}
        \biggl(\dfrac{2}{3}\biggr)^2 + h^2 = 36 \\
        \dfrac{4}{9} + h^2 = 36 \\
        h^2 = \dfrac{320}{9} \\
        h = \dfrac{8}{3}\sqrt {5}
    \end{gather*}
    
    Chiamiamo ora $K$ la proiezione del punto $M$ sulla retta $AB$ (non è segnata sulla figura).
    Abbiamo che $\overline{AK} = \overline{AB} + \overline{BH} - \overline{KH}$, con $\overline{KH} = \overline{MC} = 6$.
    Quindi:
    \begin{equation*}
        \overline{AK} = 12 + \dfrac{2}{3} - 6 = \dfrac{20}{3}
    \end{equation*}
    
    Applicando ora il teorema di Pitagora al triangolo rettangolo $AKM$ si ha:
    \begin{align*}
        \overline{AM} &= \sqrt {\overline{AK}^2 + \overline{KM}^2} = \\
        &= \sqrt {\biggl(\dfrac{20}{3}\biggr)^2 + \biggl(\dfrac{8}{3}\sqrt {5}\biggr)^2} = \\
        &= \sqrt {\dfrac{400}{9} + \dfrac{320}{9}} = \\
        &= \sqrt {\dfrac{720}{9}} = \\
        &= \sqrt {80} = \\
        &= 4\sqrt {5}
    \end{align*}
    

\end{soluzione}